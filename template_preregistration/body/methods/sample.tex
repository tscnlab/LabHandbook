\subsection{Sample}

\subsubsection{Inclusion and Exclusion Criteria}
%% [CONSORT 4a]
\begin{prereg}
\begin{instruction}
List any implicit and explicit eligibility criteria used to determine if a participant is included in the study. This could include age ranges, sex, chronotype, visual function, consumption of substances, and any participant-level characteristics precluding them from inclusion in the study. Include information about the aspect (e.g., age), assessment modality (e.g. self-report), exclusion cut-off, the timing of assessment (e.g. during screening visit, …). If only the table is used, this section needs to explain the table and what is said in the table. We recommend not duplicating information.
\end{instruction}
\end{prereg}

% Please add the following required packages to your document preamble:
% \usepackage{booktabs}
% \usepackage{graphicx}
\begin{table}[ht]
\centering
\small
\caption{Inclusion and exclusion criteria}
\label{tbl_incl_excl}

\begin{tabularx}{\linewidth}{XXXXXX}
\toprule
Aspect & Assessment modality & Exclusion criterion and cut-off                                 & Timing of Screening      \\ 
\midrule
Age    & Self-report         & \begin{tabular}[c]{@{}l@{}}< 18 years\\ > 35 years\end{tabular} & Initial screening survey \\
Gender                               &  &  &  \\
Diseases                             &  &  &  \\
Sleep disorders                      &  &  &  \\
Drug/ alcohol use                    &  &  &  \\
Body Mass Index (BMI)                &  &  &  \\
Visual acuity                        &  &  &  \\
Shift word < 3 months prior to study &  &  &  \\ 
\bottomrule
\end{tabularx}

\end{table}

\subsubsection{Participant-level Characteristics}
%% [3.1.1 CIE TN 011:2020 / 4a Consort statement]
\begin{prereg}
\begin{instruction}
Define data that is collected to describe the characteristics of the population.
\end{instruction}
\end{prereg}

\subsubsection{Participant Recruitment}
%% [14a Consort statement]
\begin{prereg}
\begin{instruction}
Method of recruitment, such as by referral or self selection (for example, through advertisements). Describe screening procedure. The steps could include: 
\begin{itemize}
    \item Telephone or physical interview (Volunteers are informed in detail about the study and questions they might have will be answered. Inclusion and exclusion criteria are screened as far as possible) 
    \item Questionnaires (e.g. consent form, General Medical questionnaire, the Horne-Östberg (Morningness-Eveningness) Questionnaire, the Pittsburgh Sleep Questionnaire (PSQI), the Epworth Sleepiness Scale (ESS), and the Beck Depression Inventory (BDI-II), Munich Chronotype Questionnaire (MCTQ)). How will the participants fill in the questionnaires (e.g. via an online-platform (REDcap)). 
    \item Physical examination
    \item Ophthalmologic Screening (e.g. Visual acuity, contrast sensitivity, colour vision (100 Hue, Ishihara, …), stereoscopic vision)
\item Adaptation night 
\end{itemize}
Provide information under which circumstances volunteers will be excluded from the study, e.g. if compliance to this outpatient segment of the study is verified using wrist actigraphs and self-reported sleep logs, what is the maximum deviation allowed. Report if and how a toxicological analysis will be carried out. Describe how volunteers will be compensated for participating in the study.
\end{instruction}
\end{prereg}


\subsubsection{Stopping guidelines}
\begin{prereg}
\begin{instruction}
Describe the condition which has to be achieved to stop data collection (e.g. number of participants). Also name adverse events which would lead to termination of the study before completion on a study level. On a participants level, state what will lead to exclusion of the participant in the course of the study.
\end{instruction}
\end{prereg}

